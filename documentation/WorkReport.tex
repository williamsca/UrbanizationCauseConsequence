\documentclass[12pt]{article}
\usepackage{hyperref}

\title{The Cause and Consequences of Urbanization}
\begin{document}

Across the world, nearly every country has experienced increasing levels of urbanization. Since 1960, the average percent of the world population living in areas classified as urban has grown from just under 34\% to nearly 58\% (World Bank). For the most part, this transition has closely tracked economic development, and relationship between the two is debated. On one hand, economic growth may drive urbanization, as rising incomes cause demand for services and manufactured goods to increase relative to that for agricultural products, thus pulling workers into cities where they can produce those goods. On the other hand, urbanization may itself increase economic growth, as urban residents use their better access to technology and greater learning opportunities to become more productive. Micro evidence for this view follows from considering the tendency of poor people move to cities rather than away, even when slum-like conditions await them.

Moreover, urbanization also has an hypothesized relationship to the quality of governance. Glaeser (2014) suggests that urban areas drive improvements in political organizations. It is easy to bring to mind historical instances of this theorized connection: both the American Revolution and the Civil Rights movement benefitted greatly from the ease of communication and coordination that proximate, urban living allowed. Conversely, weak political organizations may find it impossible to effectively police and regulate regions outside the capital city, incentivizing citizens to relocate to urban centers of power.

In order to investigate the effects of economic development and governance on the process of urbanization, this project used data available from the World Bank that stretch back to 1960 and comprise more than 180 countries. The data were tested for the presence of \textit{unit roots} and \textit{cointegration}. Both of these properties are necessary in order to apply tests for long-run causality developed by Williams professor Peter Pedroni and co-author David Canning (2008). With them, we find evidence that: (1) shocks to GDP have inverse effects on the long-run trajectory of urbanization, and (2) shocks to government effectiveness also have inverse effects on the long-run trajectory of urbanization. In short, a random increase in GDP or in the quality of governance result in lower levels of urbanization in the long run.

\begin{thebibliography}{9}

\bibitem[(2014)]{manual}Glaeser, E.L., 2014. A World of Cities. Journal of the European Economic Association, 12, 11541199.

\bibitem[(2008)]{manual}Canning, David and Pedroni, Peter, 2008. Infrastructure, Long Run Economic Growth and Causality Tests
for Cointegrated Panels. The Manchester School Vol 76 No. 5, pp 504527.

\end{thebibliography}

\end{document}